
%
% Custom chapter style
%
\makechapterstyle{evans}{%
	\renewcommand*{\chapnamefont}{\hpchap\normalsize}
	\renewcommand*{\chapnumfont}{\chapnamefont\normalsize}
	\renewcommand*{\chaptitlefont}{\hp\large}

	\setlength{\beforechapskip}{0pt}
	\setlength{\midchapskip}{1.1cm plus .5\baselineskip minus .5\baselineskip}
	\setlength{\afterchapskip}{1.1cm plus .5\baselineskip minus .5\baselineskip}% Some stretchable glue is necessary to avoid vbox fill messages

	\renewcommand{\printchapternum}{ % " C H A P T E R   O N E "
		\begin{center} \chapnumfont \hyperref[contents]{CHAPTER \NUMBERstringnum{\thechapter}}\end{center}
	}

	\renewcommand*{\printchaptername}{}% Not needed, "CHAPTER" added above

	\renewcommand{\printchaptertitle}[1]{ % "A DAY OF VERY LOW PROBABILITY"
		\begin{center}\chaptitlefont \MakeUppercase{##1}\end{center}
	}

	\renewcommand{\chaptermark}[1]{% "CHAPTER ONE" and "A DAY OF..."
		\markboth
			{\MakeUppercase{\chaptername}~\thechapter}
			{\MakeUppercase{##1}}
	}

	\renewcommand*{\tocmark}{\markboth{}{\MakeUppercase{Contents}}}

	\renewcommand{\tocheadstart}{\chapterheadstart}
	\renewcommand{\aftertoctitle}{\thispagestyle{empty}\afterchaptertitle}
}
\chapterstyle{evans}% Actually implements the above chapter style code


\makechapterstyle{omake}{%
	\renewcommand*{\chapnamefont}{\hpchap\normalsize}
	\renewcommand*{\chapnumfont}{\chapnamefont\normalsize}
	\renewcommand*{\chaptitlefont}{\hp\large}

	\setlength{\beforechapskip}{0pt}
	\setlength{\midchapskip}{0pt}
	\setlength{\afterchapskip}{1\baselineskip}

	\renewcommand*{\printchapternum}{% " O M A K E   F I L E  IV"
		\begin{center} \chapnumfont \hyperref[contents]{OMAKE FILE \Roman{chapter}}\end{center}
	}

	\renewcommand*{\printchaptername}{} % I don't want the number and name in seperate lines so  it's already defined above

	\renewcommand*{\printchaptertitle}[1]{% "THE OTHER FANFICTIONS YOU COULD HAVE BEEN READING"
		\vskip 1cm \begin{center}\chaptitlefont \MakeUppercase{##1}\end{center}\par \vskip 1cm}

	\renewcommand*{\chaptermark}[1]{% "OMAKE FILE IV" and "THE OTHER FANFICTIONS>.."
		\markboth{\MakeUppercase{##1}}{%
			\MakeUppercase{\chaptername}~\protect\NUMBERstringnum{\thechapter}}}

	\renewcommand*{\tocmark}{\markboth{}{\MakeUppercase{Contents}}}

	\renewcommand{\tocheadstart}{\chapterheadstart}
	\renewcommand{\aftertoctitle}{\thispagestyle{empty}\afterchaptertitle}
}


\newcommand*{\hpmorchapbreak}{\linebreak} % Allows the break to be disabled in TOC

\newcommand{\wrapchapter}[2]{
	\chapter[% Custom linebreak for long chapter titles
        \texorpdfstring{%
			#1\protect\hpmorchapbreak%
		#2}{% Header
        #1 #2}
	]{% Bookmark for PDF
        #1\protect\hpmorchapbreak #2}}% First page of chapter

% Command to choose where to break the line in long chapter titles
%% \newcommand*{\wrapchapter}[2]{
%% 	\chapter[#1 #2]           % TOC and page header
%% 	{#1\protect\linebreak #2}}% Title

% Macro to format an abbreviation such as "TSPE" for "the Stanford Prison Experiment"
\newcommand{\abbrev}[1]{{\scshape \MakeLowercase{#1}}}

% Formatting of the part numbers.
% Numbers are formatted as roman numerals (0 is left unchanged).
% Non-numbers are also left unchanged, and without a "Part~" prefix. This is used for chapter 77.
% This implementation is expandable and so causes no issue with hyperref.
\newcommand{\formatPart}[1]{\ifnum0<0#1\relax Part~\RomNum{#1}\else \if 0#1\relax Part~0\else #1\fi\fi}
\newcommand{\formatPartShort}[1]{\ifnum0<0#1\relax \RomNum{#1}\else #1\fi}% Same case for 0 and non-numeric

% \partchapter{The Stanford Prison Experiment}{1}
% TOC: The Stanford Prison Experiment, Part I
% Page header: The Stanford Prison Experiment I
% Title: The Stanford Prison Experiment, Part I
\newcommand{\partchapter}[2]{%
	\chapter[#1, \formatPart{#2}]%
		[#1 \formatPartShort{#2}]%
		{#1, \formatPart{#2}}}

% \namedpartchapter{Taboo Tradeoffs}{2}{The Horns Effect}
% TOC: Taboo Tradeoffs, Part II: The Horns Effect
% Page header: Taboo Tradeoffs II: The Horns Effect
% Title: Taboo Tradeoffs, Part II: \\? The Horns Effect
% The optional argument affects whether the line is broken after "Part ...:"
\newcommand{\namedpartchapter}[5][1]{%
	\chapter[#2, \formatPart{#3}: #4]%
		[\mbox{#2 \formatPartShort{#3}:} \mbox{#4}]%
		{#2, \formatPart{#3}:\protect\linebreak[#1] #4}}

% \abbrevnamedpartchapter{The Stanford Prison Experiment}{TSPE}{6}{Constrained Optimization}
% TOC: \abbrev{TSPE}, Part VI: Constrained Optimization
% Page header: TSPE VI: Constrained Optimization
% Title: The Stanford Prison Experiment, Part VI: \\? Constrained Optimization
% The optional argument affects whether the line is broken after "Part ...:"
\newcommand{\abbrevnamedpartchapter}[6][1]{%
	\chapter[\texorpdfstring{\abbrev{#3}, \formatPart{#4}: #5}{#3, \formatPart{#4}: #5}]%
		[\mbox{#3 \formatPartShort{#4}:} \mbox{#5}]%
		{#2, \formatPart{#4}:\protect\linebreak[#1] #5}}


%
% Subsection
%
\setsubsecheadstyle{\bfseries\scshape} % Subsection titles in small caps
\setsubsechook{\nopagebreak\vskip 0pt plus 3\baselineskip}
