
                                %%%%%%%%%%%%%%%%%%%%%%%%%%
                                %%     DEPENDANCIES     %%
                                %%  - hp-spacing.tex    %%
                                %%%%%%%%%%%%%%%%%%%%%%%%%%



%%%%%%%%%%%%%%%%%%%%%%%%%%%%%%%%%%%%%%%%%%%%%%%%%%%%%%%%%%%%%%%%%%%%%%%%%%%%%%%%%%%%%%%%%%%%%%%%%%%
%%                              PREFACE: LOGICAL FORMATTING & LATEX                              %%
%%                                                                                               %%
%%  We often like to indicate a specific type of text (quote, headline, shout, etc.) with        %%
%%  typographical changes to it. Instead of formatting each individual case of such text, LaTeX  %%
%%  allows us to mark lines as a certain type of text, and then define the formatting for        %%
%%  that type elsewhere -- so if we used capitalization to indicate a character is shouting,     %%
%%  but then we decide we actually want to mark it with boldface, we could simply change the     %%
%%  line that defined \shout and it would change all instances of shouting immediately.          %%
%%                                                                                               %%
%%  This file includes, then, all the definitions of the logical formatting used in the book;    %%
%%  parseltoungue, prophecy, headlines, letters, thoughts, chalkboard and inscription.           %%
%%                                                                                               %%
%%  This file is divided into two sections: inline commands, and enviornments, based on the      %%
%%  LaTeX feature they utilize.                                                                  %%
%%                                                                                               %%
%%%%%%%%%%%%%%%%%%%%%%%%%%%%%%%%%%%%%%%%%%%%%%%%%%%%%%%%%%%%%%%%%%%%%%%%%%%%%%%%%%%%%%%%%%%%%%%%%%%



%%%%%%%%%%%%%%%%%%%%%%%%%%%%%%%%%%%%%%%%%%%%%%%%%%%%%%%%%%%%%%%%%%%%%%%%%%%%%%%%%%%%%%%%%%%%%%%%%
%%                                        INLINE COMMANDS                                      %%
%%                                                                                             %%
%% The commands defined below can be used inline on a short section of text to format it based %%
%% on the type of text it represents (e.g headline, prophect, etc.).                           %%
%%                                                                                             %%
%% For example:                                                                                %%
%% This text is normal, but \parsel{thiss text isss in parseltoungue}. This text is normal     %%
%% again.                                                                                      %%
%%%%%%%%%%%%%%%%%%%%%%%%%%%%%%%%%%%%%%%%%%%%%%%%%%%%%%%%%%%%%%%%%%%%%%%%%%%%%%%%%%%%%%%%%%%%%%%%%
\newcommand*{\secondemph}[1]{\textbf{#1}} % Alternative emphasis (replaces some original       %%
                                          % capslock formatting)                               %%
                                                                                               %%
\newcommand*{\doubleemph}[1]{% Parseltongue (Italicized)                                       %%
  \emph{\secondemph{\microtypecontext{kerning=pemph}#1}}                                       %%
}                                                                                              %%
                                                                                               %%
\newcommand*{\houseverdict}[1]{% Parseltongue (Italicized)                                     %%
  \MakeUppercase{#1}                                                                           %%
}                                                                                              %%
                                                                                               %%
\newcommand*{\shout}[1]{% Parseltongue (Italicized)                                            %%
  \textit{#1}                                                                                  %%
}                                                                                              %%
                                                                                               %%
\newcommand*{\parsel}[1]{% Parseltongue (Italicized)                                           %%
  \textit{#1}                                                                                  %%
}                                                                                              %%
                                                                                               %%
\newcommand*{\prophecy}[1]{% Prophecies (Italicized, bold and capitalized)                     %%
  \textbf{\textit{\textsc{#1}}}%                                                               %%
}                                                                                              %%
                                                                                               %%
\newcommand{\cantvoice}[1]{} % For CH89 when Harry is so captured by terror he cannot even     %%
                             % bring himself to mentally voice Hermione's name. Most PDF       %%
                             % versions put a censor bar instead of completley omitting it.    %%
                                                                                               %%
                                                                                               %%
\newcommand*{\inlineheadline}[1]{% Newspaper headlines inline with text. (Capitalized)         %%
  \textsc{#1}%                                                                                 %%
}                                                                                              %%
                                                                                               %%
\newcommand{\letterAddress}[1]{% Letter Address, for use with writtenNote enviornment below.   %%
  \pagebreak[1]    % It's nice to page break here (though not a must -- that's the meaning     %%
                   % of the optional argument).                                                %%
  \noindent        % Don't indent                                                              %%
  #1               % Write the adress                                                          %%
  \nopagebreak\par % Stick the adress to the rest of the letter.                               %%
}                                                                                              %%
                                                                                               %%
\newcommand{\letterClosing}[2]% Letter Address, for use with writtenNote enviornment below.    %%
           [\vskip 1\baselineskip]{                                                            %%
  \nopagebreak[4] % Make sure it sticks to the letter above                                    %%
  #1\par          % Write the closing with an appended newline                                 %%
  \nopagebreak[5] % Maje su                                                                    %%
  \noindent#2                                                                                  %%
}                                                                                              %%
                                                                                               %%
\newcommand{\chalkboard}[1]{ % chalkboard, used in Chapter 15 for "Transfiguration is Not      %%
                             % Permanant"`                                                     %%
    \SmallVSpace%                                                                              %%
    \centerline\Large\MakeUppercase{#1}                                                        %%
    \SmallVSpace%                                                                              %%
}                                                                                              %%
%%%%%%%%%%%%%%%%%%%%%%%%%%%%%%%%%%%%%%%%%%%%%%%%%%%%%%%%%%%%%%%%%%%%%%%%%%%%%%%%%%%%%%%%%%%%%%%%%

%%%%%%%%%%%%%%%%%%%%%%%%%%%%%%%%%%%%%%%%%%%%%%%%%%%%%%%%%%%%%%%%%%%%%%%%%%%%%%%%%%%%%%%%%%%%%%%%%
%%                                        ENVIRONMENTS                                         %%
%%                                                                                             %%
%% The enviornments defined below can be used to indicate a block of text represents some type %%
%% of speech or text (e.g letter, thought) and format it apropriately; here the details of the %%
%% formatting is defined.                                                                      %%
%%                                                                                             %%
%% Enviornments are a TeX feature that allows delinating a block of text with \begin{env} and  %%
%% \end{env) to change their formatting, De facto, both \begin{env} and \end{env} are replaced %%
%% with text and commands defined together with the enviornment 'env'                          %%
%%                                                                                             %%
%% The command \newenvironment used below to define enviornments is quite simple; the first    %%
%% argument is the eviornment name, the second argument replaces each instance of its \begin   %%
%% and the third argument replaces each instance of its \end                                   %%
%%                                                                                             %%
%% Example Usage:                                                                              %%
%%                                                                                             %%
%% Not a thought                                                                               %%
%% \begin{thought}                                                                             %%
%% This is some thoughtful text                                                                %%
%% this is a thought, too                                                                      %%
%% \end{thought}                                                                               %%
%% This is no longer a thought                                                                 %%
%%%%%%%%%%%%%%%%%%%%%%%%%%%%%%%%%%%%%%%%%%%%%%%%%%%%%%%%%%%%%%%%%%%%%%%%%%%%%%%%%%%%%%%%%%%%%%%%%
                                                                                               %%
                                                                                               %%
\NewEnviron{upperCase}{\MakeUppercase{\BODY}}                                                  %%
                                                                                               %%
                                                                                               %%
\newenvironment{writtenNote}{ % Makes a written note in italics (first used, Hogwarts          %%
                              % acceptance letter)                                             %%
    \SomeVSpace       % Open with vertical space in line with grid                             %%
    \begin{adjustwidth}{\parindent}{\parindent} % Increases left/right margins symmetrically   %%
                                                % by the new paragraph indent length           %%
    \begin{em}        % Emphasis                                                               %%
        \par\noindent % newline without indentation                                            %%
        \nopagebreak[4]                                                                        %%
}{%                                                                                            %%
    \end{em}          % end emphasis                                                           %%
    \end{adjustwidth} % End margins change                                                     %%
    \SomeVSpace       % CLosr with vertical space in line with grid                            %%
}                                                                                              %%
                                                                                               %%
%% NOTE here:                                                                                  %%
%% In the original text                                                                        %%
\newenvironment{gameNote}{ % Notes for "The Game" from Ch13. Centered, Capitalized,            %%
                           % WrittenNote style defined above.                                  %%
    \begin{center}                                                                             %%
    \scshape                                                                                   %%
    \setlength{\parindent}{0pt}                                                                %%
    \setlength{\parskip}{1em}                                                                  %%
    \begin{adjustwidth}{\parindent}{\parindent} % Increases left/right margins symmetrically   %%
    \renewcommand{\emph}[1]{\uline{##1}}
    \nopagebreak[1] % Discourage pagebreaks before game note starts
}{

    \end{adjustwidth} % End margins change                                                     %%
    \end{center}
    \SomeVSpace       % Open with vertical space in line with grid                             %%
}                                                                                              %%

\newenvironment{gameNoteText}{}{} % Currently note text is not formatted

\newenvironment{gameNotePoints}{ % Notes for "The Game" from Ch13. Centered, Capitalized,      %%
    \smallskip                                                                                 %%
    \newcolumntype{U}{>{\collectcell\titlecap}r<{\endcollectcell}} % Right-aligned titlecaps   %%
    \begin{tabular}[b]{U@{: }l} % First column is titlecaps-right-aligned (see above) and ends %%
                                % with a colon. Second column is left aligned.                 %%
}{                                                                                             %%
    \end{tabular}                                                                              %%
}                                                                                              %%
                                                                                               %%
\newenvironment{onTheHeritability}{% For the CH22 "Paper"                                      %%
    \begin{writtenNote}
        \begin{center}
            \setlength{\parskip}{1\gridsmallvskip}
}{
        \end{center}
    \end{writtenNote}
}
                                                                                               %%
                                                                                               %%
\newenvironment{thought}[1]  % Format a "thought" paragraph: vertical space around, consistent %%
                             % and controllable indent                                         %%
               [\parindent]{ % Optionl argument for thought indent                             %%
    \SmallVSpace   % vertical space, also terminates previous paragraph                        %%
    \hangindent=#1 % apply indent given in optional argument                                   %%
    \hangafter=0   % apply indent from the first line                                          %%
    \parindent=0pt % disable extra indent for first line                                       %%
}{%                                                                                            %%
    \SmallVSpace% vertical space, also terminates paragraph                                    %%
}                                                                                              %%
                                                                                               %%
% Makes an inscription in small caps (first used, Gringotts)                                   %%
\newenvironment{inscription}{%\parindent=1cm%                                                  %%
    \SomeVSpace%                                                                               %%
	\begin{adjustwidth}{4\parindent}{\parindent}                                               %%
    \scshape                                                                                   %%
}{%                                                                                            %%
    \end{adjustwidth}                                                                          %%
    \SomeVSpace%                                                                               %%
}                                                                                              %%
%%%%%%%%%%%%%%%%%%%%%%%%%%%%%%%%%%%%%%%%%%%%%%%%%%%%%%%%%%%%%%%%%%%%%%%%%%%%%%%%%%%%%%%%%%%%%%%%%
