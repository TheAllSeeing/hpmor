{
    \pagestyle{empty}

    \setlength{\parindent}{0pt}
    \setlength{\parskip}{.5\baselineskip}

    \centerline{\Large{\textbf{This is your final exam.}}}

    \vspace{1.5\baselineskip}

    \emph{Your solution must at least allow Harry to evade immediate death,
    despite being naked, holding only his wand, facing 36 Death Eaters
    plus the fully resurrected Lord Voldemort.}

    Keep in mind the following:

    1. \uline{Harry must succeed via his own efforts}. The cavalry is not coming.
    Everyone who might want to help Harry thinks he is at a Quidditch game.

    2. \uline{Harry may only use capabilities the story has already shown him to have}.
    he cannot develop wordless wandless Legilimency in the next 60 seconds.

    3. \uline{Voldemort is evil and cannot be persuaded to be good}.
    the Dark Lord's utility function cannot be changed by talking to him.

    4. \uline{If Harry raises his wand or speaks in anything except Parseltongue,
    the Death Eaters will fire on him immediately}.

    5. If the simplest timeline is otherwise one where Harry dies---\uline{if
    Harry cannot reach his Time-Turner without Time-Turned help---then
    the Time-Turner will not come into play}.

    6. \uline{It is impossible to tell lies in Parseltongue}.

    Within these constraints,
    Harry is allowed to attain his full potential as a rationalist,
    now in this moment or never,
    regardless of his previous flaws.

    Of course ``the rational solution'',
    if you are using the word ``rational'' correctly,
    is just a needlessly fancy way of saying `the best solution'
    or ``the solution I like'' or ``the solution I think we should use'',
    and you should usually say one of the latter instead.
    (We only need the word ``rational'' to talk about ways of thinking,
    considered apart from any particular solutions.)

    And by Vinge's Principle,
    if you know exactly what a smart mind would do,
    you must be at least that smart yourself.
    Asking someone "What would an optimal player think is the best move?"
    should produce answers no better than "What do you think is best?"

    So what I mean in practice,
    when I say Harry is allowed to attain his full potential as a
    rationalist,
    is that Harry is allowed to solve this problem
    the way \emph{you} would solve it.
    If you can tell me exactly how to do something,
    Harry is allowed to think of it.

    But it does not serve as a solution to say, for example,
    "Harry should persuade Voldemort to let him out of the box"
    if you can't yourself figure out how.


    I wish you the best of luck, or rather the best of skill.

    (And really, truly, I do mean it,
    Harry cannot develop any new magical powers
    or transcend previously stated constraints on them
    in the next sixty seconds.)

    \sbreak%

    Harry Potter and the Methods of Rationality was originally posted as serial fiction
    on the fan forum fanfiction.net. The above text was appended to the original posting
    of chapter 113 (``Final Exam'' in this book), giving readers 60 hours to submit a
    viable solution, and stating otherwise they ``will get a shorter and sadder ending''.

    Chapter 114 was indeed posted and you may turn the page and find it. You, dear reader, can
    take as much time as you wish, or give up the task completely. Still, this \emph{is}  your
    Final Exam, and you are encouraged to take it.

    For best experience, we suggest not searching the Internet---or your friends---for ideas, though
    discussion and debate in trying to \emph{come up} with an idea is allowed; rationality can be a group effort.
}


\cleartoverso%
\cleartoverso%
